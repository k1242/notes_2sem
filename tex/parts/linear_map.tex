\section{Линейные отображения}


\subsection{Линейные отображения векторных пространств}

\begin{to_def}
    Отображение $f \colon V \to W$ называется \textit{линейным}, если
    $$
        f(\vc{x} + \vc{y}) = f(\vc{x}) + f(\vc{y}), \hspace{0.5cm} f(\lambda \vc{x}) = \lambda f(\vc{x}). 
    $$
\end{to_def}

С любым линейным отображением $f \colon V \to W$ ассоциируются два подпространства:
\begin{align*}
    \text{\textit{ядро}:}  \hspace{0.5cm} \Ker f &= \{\vc{v} \in V \mid f(\vc{v}) = 0\}, \\
    \text{\textit{образ}:} \hspace{0.5cm} \Im f &= \{\vc{w} \in W \mid \vc{w} = f(\vc{v}), \vc{v} \in V\}.
\end{align*}

\begin{to_thr} 
    Пусть $V$ над $\mathbb{F}$, $f \colon V \to W$. Тогда $\Ker f$, $\Im f$ конечномеры и 
    $$
        \dim \Ker f +\dim \Im f = \dim V.
    $$
\end{to_thr}

\begin{proof}[$\triangle$]
    Так как $\Ker f \subset V$, то $\dim \Ker f \leq \dim V \leq \infty$. Любой вектор из $\Im f$ имеет вид
    $$
        f\left( \sum\nolimits_{i=1}^n \alpha_i \vc{e}_i \right) = \sum\nolimits_{i=k+1}^n \alpha_i f(\vc{e}_i), \hspace{0.5cm} \alpha_i \in \mathbb{F}.
    $$
    т.е. векторы $f(\vc{e}_{k+1}), \ldots, f(\vc{e}_{n})$ порождают $\Im f$. 

    Они линейно независимы. Действительно, пусть $\sum_{i=k+1}^n \lambda_i f(\vc{e}_i) =  0$. Тогда $f(\sum_{i=k+1}^n \lambda \vc{e}_i) =0$. Это значит, что $\sum_{i=k+1}^n \lambda_i \vc{e}_i \in \Ker f$. Но всякая линейная зависимость между базисными элементами должна быть тривиальной.
\end{proof}




\subsection{Аффинные (точечные) пространства}

Во-первых в этом параграфе введем множество \textit{точек} $\dot p, \dot q, \dot r, \ldots$. Назовём его $\mathbb{A}$. Пусть $V$ -- векторное пространство над $\mathbb{F}$. Пара $(\mathbb{A}, V)$  называется \textit{аффинным пространством}, ассоциированным (или связанным) с $V$, если задано отображение $(\dot p, \vc{v}) \to \dot p + \vc{v}$, такое, что:

1) $\dot p + \vc{0} = \dot p$, $(\dot p + \vc{u}) + \vc{v} = \dot p + (\vc{u} + \vc{v})$ для $\forall p \in \mathbb{A}$ и $\forall \vc{u}, \vc{v} \in V$;

2) $\forall \dot p, \dot q \in \mathbb{A}$, $\exists ! \vc{v} \in V \colon \dot p + \vc{v} = \dot q$.


\begin{to_def} 
    Пусть $\mathbb{A}, \mathbb{A}'$  -- аффинные пространства, ассоциированные с векторными пространствами $V, V'$ над одним и тем же $\mathbb{F}$. Отображение $f \colon \mathbb{A} \to \mathbb{A}'$ называется \textit{аффинным} (или \textit{аффинно-линейным}), если $\forall \dot p \in \mathbb{A}, \; \vc{v} \in V$ выполнено соотношение 
    \begin{equation}
        f(\dot p + \vc{v}) = f(\dot p) + D f \cdot \vc{v},
    \end{equation}
    где $Df \colon V \to V'$ -- линейное отображение векторных пространств. Отображение $Df$ называют иногда \textit{линейной частью} (или \textit{дифференциалом}) отображения $f$. Для биективного аффинно-линейного отображения $f$ линейная часть $Df$ тоже биективна. В этом случае говорят об изоморфизме между $\mathbb{A}$ и $\mathbb{A}'$, а при $\mathbb{A}' = \mathbb{A}$ -- об \textit{аффинном автоморфизме} пространства $\mathbb{A}$, реализованном посредством \textit{невырожденного аффинного преобразования} $f$.
\end{to_def}

Из такого определения становится очевидным такой ряд свойств, как сохранение параллельности, отношения между отрезками и т.д. связанного с биективностью отображения. Примером таких преобразований служит поворот, растяжение/сжатие, отражение, перенос.

\begin{to_def} 
    \textit{Системой координат} в $n$-мерном аффинном пространстве $(\mathbb{A}, V) $ называется совокупность $\{\dot o; \vc{e}_1, \ldots, \vc{e}_n\}$ точки $\dot o \in \mathbb{A}$  и базиса $(\vc{e}_1, \ldots, \vc{e}_n)$ в $V$. Координатами $x_1, \ldots, x_n$ точки $\dot p$ считаются координаты вектора $\overline{op}$ в базисе $(\vc{e}_1, \ldots, \vc{e}_n)$: $\overline{op} = x_1 \vc{e}_1 + \ldots + x_n \vc{e}_n$.
\end{to_def}

\begin{to_def} 
     Пусть $\dot p$ -- фиксированная точка $n$-мерного аффинного пространства $(\mathbb{A}, V)$ и $U$ -- векторное подпространства в $V$. Тогда множество
     $$
         \Pi = \dot p + U = \{\dot p + \vc{u} \mid \vc{u} \in U\}
     $$
     называется \textit{плоскостью} (или \textit{афинным подпространством}) в $\mathbb{A}$ размерности $m  =\dim U$. Считается, что $\Pi$ проходит через точку $\dot p$ в направлении $U$. 
\end{to_def}

Проведём некоторое рассуждения, для понимания необходимости этого языка. Пусть $\dot q = \dot p + \vc{u}$, $\dot r = \dot p + \vc{v}$, \; $\vc{u}, \vc{v} \in U$, то
$$
    \dot q + (\vc{v} - \vc{u}) = \dot p + \vc{u} + (\vc{v} - \vc{u}) = \dot p + \vc{v} = \dot r.
$$
Тогда $\overline{qr} = \vc{v} - \vc{u}$, соответственно из $\dot q, \dot r \in \Pi \Longrightarrow \overline{qr} \in U$.

\begin{to_thr} 
    Всякая плоскость $\Pi = \dot p + U$  в аффинном пространстве сама является афинным пространством, ассоциированным с $U$.
\end{to_thr}

\begin{to_thr} 
% стр 178
    Подмножество $\Pi \subset \mathbb{A}$ тогда, и только тогда является подпространством, когда оно целиком содержит прямую, проходящую через любые две его различные точки.
\end{to_thr}

\begin{to_def} 
    Любае две плоскости в направлении одного и того же подпространства $U$ называют параллельными.
\end{to_def}

Аналогично можно определить аффинный функционал. Отображение $f \colon \mathbb{A} \to \mathbb{F}$ называется аффинно-линейной функцией, если
$$
    f(\dot p + \vc{v}) = f(\dot p) + D f \cdot \vc{v} \hspace{0.5cm} 
    \forall \dot p \in \mathbb{A}, \vc{v} \in V.
$$

Выбрав систему координат $\{\dot o; \vc{e}_1, \ldots, \vc{e}_n\}$, выразим значение $f$ в виде
$$
    f(\dot p) = f(\dot o + \overline{op}) = \sum_{i=1}^n \alpha_i x_i + \alpha_0,
$$
где $\alpha_0 = f(\dot o), \alpha_i = D f \cdot \vc{e}_i$, $\overline{op} = x_1 \vc{e}_1 + \ldots + x_n \vc{e}_n$.


\begin{to_thr} 
    Пусть $\mathbb{A}$ -- аффинное пространство размерности $n$. Множество точек из $\mathbb{A}$, координаты которых удовлетворяют совместной системе линейных уравнений ранга $r$, образуют $(n-r)$-мерную плоскость $\Pi \subset \mathbb{A}$. Любая плоскость в $\mathbb{A}$ может быть так получена.
\end{to_thr}

\begin{to_def} 
    Пусть $\Pi' = \dot p + U'$, $\Pi'' = \dot q + U''$ ($U'$, $U''$ --- векторные подпространства в $V$ размерностей $k, l$). Говорят, что плоскость $\Pi'$ \textit{параллельна} $\Pi''$, если $U'' \subseteq U'$.
\end{to_def}

\subsection{Евклидовы (точечные) пространства}

\begin{to_def} 
    Аффинное пространство $(\mathbb{E}, V)$ называется \textit{евклидовым (точечным) пространством}, если $V$ -- евклидово векторное пространство. Или, тройка $(\mathbb{E}, V, \rho)$. 
\end{to_def}

Аналогично раннему, можем посмотреть на расстояния между объектами (см. стр. 191, $\mathbb{K}$).

\begin{to_thr} 
    Определитель Грама системы векторов $\vc{e}_1, \ldots, \vc{e}_m$, отличен от нуля в точности тогда, когда векторы системы линейно независимы. Всегда выполнено неравенство $G(\vc{e}_1, \ldots \vc{e}_m) \geq 0$. 
\end{to_thr}

\begin{to_thr} 
% стр 204
    При аффинном преобразовании $n$-мерного евклидова пространства объём параллелепипеда, построенного на $n$ векторах, умножается на модуль определителя преобразования. Другими словами, отношение объёмов параллелепипедов сохраняется. 
\end{to_thr}

\begin{to_thr} 
% стр 205
    Всякое невырожденное аффинное преобразование $f$  $n$-мерного евклидова пространства $(\mathbb{E}, V)$ есть произведение:

    1) сдвига на некоторый вектор; 

    2) движения, оставляющего неподвижной некоторую точку $\dot o$;

    3) аффинного преобразования $h$, являющегося композицией $n$ сжатий вдоль взаимно перпендикулярных осей, пересекающихся в точке $\dot o$.
\end{to_thr}

% параграф 4 с индефинитной метрикой пока будет опущен. :(