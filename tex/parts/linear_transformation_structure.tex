\section{Структура линейного преобразования}

\subsection{Алгебра линейный операторов}
При $V = W$ элемент векторного пространства $\mathcal L (V)$ называют \textit{линейным оператором} или \textit{линейным преобразованием}. 

Примерами являются: нулевой оператор $\mathcal O$ (переводит любой вектор $\vc{v} \in V$  в нулевой), оператор проектирования ($\mathcal P^2 = \mathcal P$), оператор подобия, дифференцирования, ...


\subsection{Алгебра операторов}


Отдельный интерес представляет \textbf{алгебра операторов}. Понятно, что $\mathcal L (V)$ -- векторное пространство размерности $\dim \mathcal L (V) = \left(\dim V\right)^2$. Можно по аксиомам проверить, что $\mathcal L (V)$ является одновременно  векторным пространством над $\mathbb{F}$.

\begin{to_def} 
    Кольцо $K$ является одновременно векторным пространством над $\mathbb{F}$ таким, что $\lambda(ab) = (\lambda a) b = a (\lambda b)$ для всех $\lambda \in \mathbb{F}, \; a,b \in K $, называется \textit{алгеброй} над $\mathbb{F}$. Размерность $K  $ как векторного пространства называется \textit{размерностью алгебры} $K $ над $\mathbb{F} $. Всякое векторное подпространство $L \subset K $, замкнутое относительно операции умножения в $K (L \cdot L \subseteq L)$, называется \textit{подалгеброй} алгебры $K $.
\end{to_def}

Нам интересна алгбра $\mathbb{F}[\mathcal A] $ -- наименьшая алгебра, содержащая $\mathcal A $. Какова её рзмерность? Далее докажем, что 
$$
    \dim \mathbb{F}[\mathcal A] \leq \dim V.
$$

\begin{to_def} 
    Многочлен $f(t)$ \textit{аннулирует} линейный оператор $\mathcal A$, если $f(\mathcal A) = \mathcal O $. Нормализованный многочлен минимальной степени, аннулирующий $\mathcal A $, называется \textit{минимальным многочленом} оператора $\mathcal A $. 
\end{to_def}


\begin{to_thr} 
    Для всякого линейного оператора $\mathcal A $ существует $\mu_{\mathcal A} (t) $. Оператор $\mathcal A $ обратим тогда, и только тогда, когда свободный слен $\mu_m $ отличен от нуля. 
\end{to_thr}

\begin{proof}[$\triangle$]
    Эксплуатируем тот факт, что делители нуля необратимы.
\end{proof}

\begin{to_thr} 
    Любой аннулирующий многочлен $f(t) $ оператора $\mathcal A $ делится без остатка на $\mu_{\mathcal A} (t) $. 
\end{to_thr}

\begin{to_def} 
    Линейный оператор $\mathcal A $ называется \textit{нильпотентным}, если $\mathcal A^m = \mathcal O $ для некоторого $m > 0 $; наименьшее такое натуральное число $m $ называется \textit{индексом нильпотентности}. 
\end{to_def}


\subsection{Инвариантные подпространства и собственные векторы}


\subsubsection{Проекторы}

Пусть $V = W_1 \oplus \ldots \oplus W_m $, тогда $\vc{x} \in V $:
$$
    \vc{x} = \vc{x}_1 + \ldots + \vc{x}_m, \hspace{0.5cm} x_i \in W_i,
$$
а отображение $\mathcal P_i \colon \vc{x} \mapsto \vc{x}_i \in \mathcal L (V)$. Наконец,
\begin{align*}
     W_i = \mathcal P_i V = \{\vc{x} \in V \mid \mathcal P_i \vc{x} = \vc{x}\}, \\
     K_i = \Ker \mathcal P_i = W_1 + \ldots + W_m
\end{align*} 
и $\mathcal P_i $ по сути оператор проектирования $V $ на $W_i $ вдоль $K_i $.

\begin{to_thr} 
% стр 74-75
    $\mathcal P_1, \ldots, \mathcal P_m \colon V \to V $ -- конечное множество линейных операторов таких, что 
$$
    \sum_{i=1}^m \mathcal P_i = \mathcal E; \hspace{0.5cm} \mathcal P_i^2 = \mathcal P_i, \; 1 \leq i \leq m; \hspace{0.5cm} \mathcal P_i \mathcal P_j = \mathcal O, \; i \neq j.
$$
Тогда 
$$
    V = W_1 \oplus \ldots \oplus W_m, \text{ где } _i = \Im \mathcal P_i.
$$
\end{to_thr}

\begin{proof}[$\triangle$]
    Через разбиение $\forall \vc{x} \in V $ получим
$$
    \vc{x} = \mathcal E \vc{x} = \sum \mathcal P_i \vc{x} = \vc{x}_i + \ldots + \vc{x}_m, \hspace{0.5cm} \vc{x}_i \in W_i,
$$
тоесть $V = W_1 + \ldots + W_m$. Докажем, что сумма прямая. Пусть $\vc{x} \in W_j \cap \left(\sum_{i \neq j} W_i\right) $. Но, $\exists \vc{x}_1, \ldots, \vc{x}_m $:
$$
    x = \mathcal P_j (\vc{x}_j) = \sum_{i \neq j} \mathcal P_i (\vc{x}_i).
$$
Применим $\mathcal P_j $, получим
$$
    \vc{x} = \mathcal P_j^2 (\vc{x}_j) = \sum_{i \neq j} \mathcal P_j \mathcal P_i (\vc{x}_i) = \vc{0}.
$$
\end{proof}



\subsubsection{Инвариантные подпространства}

\begin{to_def} 
    Подпространство $U \subset V $ \textit{инвариантно} относительно $\mathcal A \colon V \to V$, если $\mathcal A U \subset U $. 
\end{to_def}

\begin{to_thr} 
    Пространство $V $ является прямой суммой двух подпространств $U, W $, инвариантных относительно $\mathcal A \colon V \to V $, тогда, и только тогда, когда $\exists $ базис такой, что $\mathcal A $ принимает вид
$$
    A = \begin{pmatrix}
        A_1 & 0 \\
        0 & A_2
    \end{pmatrix}
$$
\end{to_thr}


\subsubsection{Собственные векторы. Характеристический многочлен.}

\begin{to_def} 
    Любой ненулевой вектор из одномерного подпространства, инвариантного относительно $\mathcal A $, называется \textit{собственным вектором} оператора $\mathcal A $. Если $\vc{x} $ -- собственный вектор, то $\mathcal A \vc{x} = \lambda \vc{x} $, $\lambda \in \mathbb{F} $ называется \textit{собственным значением} $\mathcal A $. 
\end{to_def}

Очевидная импликация
$
     \mathcal A \vc{x} = \lambda \vc{x}, \; \mathcal A \vc{y} = \lambda \vc{y} \Longrightarrow \mathcal A(\alpha \vc{x} + \beta \vc{y}) = \lambda(\alpha \vc{x} + \beta \vc{y})
 $ 
 даёт основание называть $V^{\lambda} $ \textit{собственным подпространством} оператора $\mathcal A $, ассоциированным с $\lambda $. Его размерность $\dim V^{\lambda} $ называется \textit{геометрической кратностью} $\lambda $.

Уместно ввести понятие \textit{характеристического многочлена}, ассоциированного с 
$\mathcal A $.  Кратность $\lambda $ как корня характеристического многочлена $\xi_{\mathcal A} (t)$ называется \textit{алгебраической кратностью} $\lambda $ оператора $\mathcal A $.

\begin{to_thr} 
     Геометрическая кратность $\lambda $ не превосходит его алгебраической кратности.
\end{to_thr}
 
\begin{proof}[$\triangle$]
Действительно, пусть $\mathcal A' $ -- ограничение $\mathcal A $ на $V^{\lambda} $, тогда $\det (t \mathcal E' - \mathcal A') = (t - \lambda)^m $, причём $\xi_{\mathcal A} (t) = (t-\lambda)^m q(t)$. Пусть $\lambda $ -- корень кратнсоти $k $ многочлена $q(t) $. Тогда алгебраической кратностью $\lambda $ будет $m+k $.
\end{proof}


\subsubsection{Критерий диагонализируемости}







\subsubsection{Существование инвариантных подпространств}
\subsubsection{Сопряженный линейный оператор}
\subsubsection{Фактороператор}