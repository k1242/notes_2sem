\section*{Сходимость и равномерная сходимость функциональных рядов}

\begin{to_def}
    Последовательность функций $f_n \colon X \to \mathbb{R}$ \textbf{сходится равномерно} на множестве $X$ к функции $f_0$, если
    $$
\sup \{
 |f_n(x) - f_0(x)| \mid x \in X   
\} \to 0, \; \; \; n \to \infty.
    $$ Также обозначается как $f_n \rightrightarrows_X f_0$.
\end{to_def}    

\begin{to_def}
    Ряд из функций $u_n \colon X \to \mathbb{R}$ \textbf{сходится равномерно}\footnote{
        Или, $\forall \varepsilon > 0 \; \exists N  \; \forall n > N \; \forall x \in E \colon |S_n(x) - f(x)| < \varepsilon$.
    } на $X$, если последовательность его частичных сумм
    $$
        S_n(x) = \sum_{k=1}^{\infty} u_k (x)
    $$
    сходится равномерно на $X$ к некоторой функции $S \colon X \to \mathbb{R}$.
\end{to_def}


\noindent
\textbf{\textit{Критерий Коши}} (равномерной сходимости \textbf{последовательности функций}):

Для последовательности функций $f_n \colon X \to \mathbb{R}$ выполняется
$$
\forall \varepsilon > 0 \; \exists N (\varepsilon)  \; \forall n, m \geq N(\varepsilon), \; \sup\{
|f_n(x) - f_m(x)| \colon x \in X
    \} < \varepsilon
$$
тогда, и только тогда, когда последовательность равномерно на $X$ сходится к некоторой функции.

%%%%%%%%%%%%%%%%%%%%%%%%%%%%%%%%%%%%%%%%%%%%%%%%%%%%%%%%%%%%%%%%%%%%%%%%

\phantom{42}

\noindent
\textbf{\textit{Критерий Коши}} (равномерной сходимости \textbf{функционального ряда}):

Для функционального ряда $\sum_{n=1}^{\infty} f_n \colon X \to \mathbb{R}$ выполняется
$$
\forall \varepsilon > 0 \; \exists N (\varepsilon)  \; \forall n, m \colon m \geq n > N(\varepsilon), \; \forall x \in E :
 \bigg|\sum_{k=n}^m f_k (x) \bigg| < \varepsilon.
$$
тогда, и только тогда, когда функциональный ряд   сходится на $X$ равномерно к некоторой функции.

%%%%%%%%%%%%%%%%%%%%%%%%%%%%%%%%%%%%%%%%%%%%%%%%%%%%%%%%%%%%%%%%%%%%%%%%

% \noindent
% \# необходимое условие сходимости функционального ряда ($\mathbb{K}$: с. 102).

\begin{to_thr}[Признак Вейерштрасса]
\textbf{Если} ряд из функций $u_n : X \to \mathbb{R}$, $\sum_{n=1}^\infty u_n(x)$, $\forall x \in X \forall n \in \mathbb{N}, |u_n(x)| \leq a_n$ и $\sum_{n=1}^\infty a_n < + \infty $, \textbf{то} ряд из функций сходится равномерно и абсолютно на $X$.

\end{to_thr}

%%%%%%%%%%%%%%%%%%%%%%%%%%%%%%%%%%%%%%%%%%%%%%%%%%%%%%%%%%%%%%%%%%%%%%%%

\begin{to_thr}[Непрерывность равномерного предела непрерывных функций] 
    \textbf{Пусть} последовательность $f_n \colon X \to \mathbb{R}$ сходится равномерно на $X$ к функции $f$ и все функции $f_n$ непрерывны. \textbf{Тогда} $f$ тоже непрерывна.
\end{to_thr}

\begin{to_thr}
    \textbf{Пусть} последовательность дифференцируемых функций $f_n \colon [a, b] \to \mathbb{R}$ сходится в точке $x_0$, а последовательность производных $f'_n$ сходится равномерно на $[a, b]$ к функции $g$. \textbf{Тогда} $(f_n)$ равномерно сходится к некоторой $f$ и $f'=g$ на всём отрезке $[a, b]$.
\end{to_thr}     