\sbs{Приближение кусочно линейными функциями и многочленами}

\begin{to_lem}
	Для непрерывной с компактным носителем $f(x) \colon \mathbb{R} \to \mathbb{R}$ и $t_n \to 0$ при $n \to \infty$, последовательность $f_n(x) = f(x + t_n) \rr f$.
\end{to_lem}

\begin{to_lem}
	$f(x) = \sqrt{x}$ можно равномерно приблизить многочленами на любом отрезке $[0,a]$.
\end{to_lem}

\begin{to_lem}
	$f(x) = |x|$ можно равномерно приблизить многочленами на любом отрезке $[-a,a]$.
\end{to_lem}

\begin{to_thr}
	Всякую непрерывную кусочно-линейную на отрезке $[a, b]$ функцию можно сколь угодно близко равномерно приблизить многочленом.
\end{to_thr}

\begin{to_lem}
	Для непрерывной $f \colon [0,1] \to \mathbb{R}$: $\sum\limits_{k = 0}^{m} f(k/m) \varphi_{1/m} (x - k/m) \rr f$.
\end{to_lem}

\begin{to_thr}
	Всякую $f \colon [a_1, b_1] \times [a_n, b_n] \to \mathbb{R}$ можно сколь угодно близко
равномерно приблизить многочленом.
\end{to_thr}

\sbs{Приближение тригонометрическими многочленами}

\begin{to_thr}[Теорема Вейерштрасса]
	Всякую непрерывную $2\pi$-периодичную $f \colon \mathbb{R} \to \mathbb{R}$ можно сколько угодно точно равномерно приблизить $T(x) = a_0 + \sum\limits_{k = 1}^{n}(a_k cos(k x) + b_k sin(k x)) $.

\end{to_thr}

\begin{to_def}
	$\mathcal{A} \subseteq C(x)$(-- непрерывные на компакте функции) называется \textit{алгеброй}, если она содержит константы ($\mathbb{R} \subseteq \mathcal{A}$) и топологически "замкнута" относительно операций $+$ и $\bullet$.
\end{to_def}

\begin{to_def}
	\textit{Алгебра разделяющая точки} --- $\forall a, b \in \mathbb{R},\, x=y \in X,\, \, \exists f \in A$ такая что $f(x)=a$, а $f(y) = b$.
\end{to_def}

\begin{to_thr}[Стоуна-Вейерштрасса]
	Если $X$-метрический компакт, а алгебра $\mathcal{A}\in C(x)$ разделяет точки, \textbf{то} $\forall f \in C(x)$ можно сколь угодно точно равномерно приблизить функциями из $\mathcal{A}$.
\end{to_thr}