\sbs{Измеримые по Лебегу и борелевские функции}

\begin{to_def}
    Функция $f \colon X \to \mathbb{R}$ называется \textbf{измеримой по Лебегу}, если для любого $c \in \mathbb{R}$ множество\footnote{
        То же верно для $f(x) \leq c$.
    } $\{ x \mid f(x) < c \}$ измеримо по Лебегу.
\end{to_def}

\begin{to_lem}
\label{lem5.39}
    Определение измеримой функции с $<$ эквивалентно $> , \leq, \geq$.
\end{to_lem}

\begin{to_thr}
\label{thr5.41}
    Поточечное взятие точной верхней или нижней грани последовательности функций не выводит за класс измеримых функций с возможно бесконечными значениями. Поточечный переход к (верхнему или нижнему) пределу также не выводит за класс измеримых функций с возможно бесконечными значениями.
\end{to_thr} 

\begin{to_def}
    \textbf{Борелевским множеством} в $\mathbb{R}^n$ называется множество, которое можно получить из открытых множеств операциями и разности множеств, счётного объединения и счётного пересечения, а также повторениями этих операций несколько раз.
Борелевской функцией называется функция\footnote{
    Сказанное выше об измеримых по Лебегу функциях относится и к борелевским, их
класс замкнут относительно арифметических операций, перехода к точной грани счётного семейства функций или к поточечному пределу.
}
, у которой все множества $\{ f(x) < c \}$ борелевские.
\end{to_def}


\begin{to_thr}
\label{thr5.45}
    Всякое измеримое $X \in \mathbb{R}^n$ можно представить в виде объединения борелевского множества и множества меры нуль. Ко всякому измеримому множеству $X \in \mathbb{R}^n$ можно добавить множество меры нуль так, что в объединении получится борелевское множество.
\end{to_thr}

\begin{to_thr}
\label{thr5.46}
    Всякую измеримую функцию $X \colon X \to \mathbb{R}$ можно переопределить на множестве меры нуль (возможно изменив область определения на меру нуль) так, что она станет борелевской.
\end{to_thr}

\begin{to_thr}
\label{thr5.48}
    Функция $f \colon X \to \mathbb{R}$ борелевская тогда, и только тогда, когда пробраз любого борелевского множества $Y \subseteq \mathbb{R}$ тоже является борелевским.
\end{to_thr}

\begin{to_def}
    Отображение $f \colon X \to Y$ метрических или топологических пространств называется \textbf{борелевским}, если прообраз всякого борелевского множества борелевский.
\end{to_def}

\begin{to_thr}
\label{thr5.50}
    Композиция борелевских отображений тоже будет борелевской.
\end{to_thr}

