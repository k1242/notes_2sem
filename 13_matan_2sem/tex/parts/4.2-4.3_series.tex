\sbs{Суммирование абсолютно сходящихся рядов}

\begin{to_def}
	\textit{Сумма ряда} --- $\sum\limits_{n = 1}^{\infty} a_n = \lim\limits_{n \to \infty} \sum\limits_{k = 1}^{n} a_k$.
\end{to_def}

\begin{to_lem}
	Если $\forall n$ $a_n \geq 0$, то $\sum\limits_{  n = 1 }^{  \infty } a_n  = \sup\left\{ \sum\limits_{ k=1 }^{ n }   a_k \mid \forall n \in \mathbb{N}\right\}$.
	\label{4.17}
\end{to_lem}

\begin{to_con}
	 Сумма ряда из чисел одного знака не меняется при перестановке её элементов.
\end{to_con}

\begin{to_def}
	Сумма ряда $\sum\limits_{ n = 1 }^{ \infty } a_n $ \textit{абсолютно сходится}, если сходится ряд $\sum\limits_{ n = 1 }^{ \infty }|a_n|$.
\end{to_def}

\begin{to_lem}[Теорема Фубини]
	Пусть сумма $\sum\limits_{ n,m = 1 }^{ \infty } a_{n m}$ сходится абсолютно, тогда $\sum\limits_{ n,m = 1 }^{ \infty } a_n = \sum\limits_{ n=1 }^{ \infty } \lr{\sum\limits_{ m = 1 }^{ \infty } a_{n m}}$.
	\label{4.21}
\end{to_lem}

\begin{to_thr}
	Если суммы рядов последовательностей $a_n$ и $b_n$ абсолютно сходятся, то:
	\begin{equation*}
	\begin{split}
		\lr{\sum\limits_{ n = 1 }^{ \infty } a_n} \cdot \lr{\sum\limits_{ m = 1 }^{ \infty } b_m} = 
		\sum\limits_{ n,m = 1 }^{ \infty } a_n b_m =
		\sum\limits_{ s = 2 }^{ \infty } \lr{\sum\limits_{ n = 1 }^{ s - 1 } a_n b_{s-n}}.
	\end{split}
	\end{equation*}
	\label{4.22}
\end{to_thr}

\begin{to_thr}
	Если  $a_n = \text{O}(b_n)$ при $n \to \infty$, то из абсолютной сходимости ряда из $b_n$ следует абсолютная сходимость ряда из $a_n$.
	\label{4.25}
\end{to_thr}

\begin{to_thr}
	Пусть $\varlimsup\limits_{n \to \infty} \sqrt[n]{a_n} = q \in [0, +\infty]$, тогда при $q > 1$ ряд из $a_n$ расходится, при $q < 1$ сходится.
\end{to_thr}

\sbs{Равномерная сходимость функциональных последовательностей}

\begin{to_def}
	$f_n \colon X \to \mathbb{R}$ $\rr f_0$ (\textit{сходится равномерно}), если $\sup\{|f_n(x) - f_0(x)| \mid x \in X\} \to 0, \, n \to \infty$.
\end{to_def}

\begin{to_def}
	Ряд из $u_n \colon X \to \mathbb{R}$ \textit{сходится равномерно на $X$}, если $S_n(x) = \sum\limits_{ k=1 }^{ n } u_k(x) \rr S$, $(S \colon X \to \mathbb{R})$.
\end{to_def}

\begin{to_thr}[Критерий Коши]
	$f_n \rr f_0$ $\Leftrightarrow$ $\forall \varepsilon > 0 \, \exists N(\varepsilon) \colon \forall n,m \geq N(\varepsilon) \leadsto \sup\{|f_n(x) - f_m(x)| \mid x \in X\} < \varepsilon$.

	\label{4.36}
\end{to_thr}

\begin{to_con}[Необходимое условие]
	Для $\sum\limits_{ n = 1 }^{ \infty } u_n(x) \rr$, \textbf{необходимо}, чтобы $u_n \rr 0$ при $n \to \infty$.	
\end{to_con}

\begin{to_thr}[Признак Вейерштрасса]
	Если $\forall x \in X\, \forall n \in \mathbb{N} \leadsto |u_n(x)| \leq a_n$ \textbf{и} $\sum\limits_{ n = 1 }^{ \infty } a_n < +\infty$ $\Rightarrow$ $\sum\limits_{ n = 1 }^{ \infty } |u_n| \rr$ 
	\label{4.39}
\end{to_thr}

\begin{to_thr}
	Пусть $f_n \colon X \to \mathbb{R} \rr f$ и все $f_n$ непрерывны, тогда $f$ тоже непрерывна.
	\label{4.40}	
\end{to_thr}

\begin{to_thr}
	\begin{equation*}
		\left.
		\begin{aligned}
		    \text{дифференцируемые } &f_n \colon [a,b] \to \mathbb{R} \text{ сходятся в }x_0\\
		    \text{последовательность }&f_n' \text{ сходится равномерно на }[a,b] \text{ к } g
		\end{aligned}
		\right\}
		\Longrightarrow
		(f_n) \underset{[a,b]}{\rightrightarrows} f\, (f' = g).
	\end{equation*}
	\label{4.41}
\end{to_thr}

\begin{to_con}
	\begin{equation*}
	\left.
	\begin{aligned}
		&\sum\limits_{ n = 1 }^{ \infty } u_n'(x) \rr \\
		\exists x_0 \in X: &\sum\limits_{ n = 1 }^{ \infty } u_n(x_0) < +\infty
	\end{aligned}
	\right\}
	\Rightarrow
	\sum\limits_{ n = 1 }^{ \infty } u_n(x) \rr \textbf{ и } \lr{\sum\limits_{ n = 1 }^{ \infty } u_n(x)}' = \sum\limits_{ n = 1 }^{ \infty } u_n'(x). 
	\end{equation*}
\end{to_con}
