\sbs{Внешняя мера Лебега и её свойства}

\begin{to_def}
    \textbf{Внешняя мера Лебега} множества $X \subset \mathbb{R}^n$ --- это точная нижняя грань по всем счётным покрытиям $X$ элементарными множествами
    $$
    \mu^* \lr{X} = \inf \left\{
    \sum_{k=1}^{\infty} \mu \lr{S_k} \mid X \subseteq  \bigcup_{k=1}^{\infty} S_k
    \right\}.
    $$.
\end{to_def}

% ТО ЖЕ ЧТО И МЕРА ЖОРДАНА ДЛЯ ЭЛЕМЕНТАРНЫХ МНОЖЕСТВ
\begin{to_lem}
\label{lem5.9}
    Для элементарного $S \subset \mathbb{R}^n$ выполняется $\mu^* (S) = m (S)$ .
\end{to_lem}

\begin{to_lem}[Счётная субаддитивность внешней меры Лебега]
\label{lem5.10}
    Для любого счётного сечения множеств $X_k \subseteq \mathbb{R}^n$, $k \in \mathbb{N}$, выполняется 
    $$
        \mu^* \lr{\bigcup_{k=1}^{\infty} X_k} \leq \sum_{k=1}^{\infty} \mu^* (X_k).
    $$
\end{to_lem}

