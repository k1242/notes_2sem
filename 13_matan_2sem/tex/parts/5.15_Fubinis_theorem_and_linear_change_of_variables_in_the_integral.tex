\sbs{Теорема Фубини и линейная замена переменных в интеграле}

\begin{to_lem}
\label{5.103}
    Всякая измеримая неотрицательная $f \colon X \to \mathbb{R}$ является счётной суммой характеристических функций измеримых по Лебегу множеств с неотрицательными коэффициентами. Если $f$ борелевская, то и характеристические функции борелевские.
\end{to_lem}

\begin{to_thr}[Теорема Фубини]
    \label{thr_fub} %104
    Пусть функция $f \colon X \times Y \to \mathbb{R}$ интегрируема на произведении интегрируемых множеств. Тогда\footnote{
        И, в частности, интеграл в скобках справа существует при $\forall x \in X$ и является измеримой функцией от $x \in X$.
    }
    $$
    \int_{X \times Y} f(x, y) \d x \d y = \int_X \lr{
    \int_Y f(x, y) \d y
    } \d x.
    $$
\end{to_thr}

% ДОКАЗАТЬ?.

\begin{to_thr}
\label{xyxy} %105
    Если множество $X \subseteq \mathbb{R}^n$ измеримо и $Y \subseteq \mathbb{R}^m$ измеримо, то $X \times Y \subseteq \mathbb{R}^{n + m}$ тоже имеримо и 
    $$
        \mu(X \times Y) = \mu (X) \cdot \mu (Y).
    $$
\end{to_thr}

% ВАЖНО
\begin{to_thr}
\label{order_hor} % 
    Пусть функция $f \colon X \to \mathbb{R}$ неотрицательна и измерима. Обозначим
    $$
        g(y) = \mu\{
        x \in X \mid f(x) \geq y
        \}
    $$
    Тогда оказывается
    $$
    \int_X f(x) \d x = \int_0^{\infty} g(y) \d y.
    $$
\end{to_thr}

\begin{to_thr}[\href{https://youtu.be/3P5iDswTCKM?t=2268}{Теорема о линейной замене переменных в интеграле Лебега}]
\label{lin_var}
    Для интегрируемой $f \colon \mathbb{R}^n \to \mathbb{R}$ и линейного преобразования $A \colon \mathbb{R}^n \to \mathbb{R}^n$
    $$
    \int_{\mathbb{R}^n} f(\vc{y}) \d \vc{y} = | \det A | \int_{\mathbb{R}^n} f(A\vc{x}) d\vc{x}.
    $$
\end{to_thr}