\sbs{Приближение интегрируемых функций в среднем}

\begin{to_def}
    Функция $f \colon \mathbb{R}^n \to \mathbb{R}$ называется \textbf{элементарно ступенчатой}, если она является ступенчатой с конечным числом ступенек, каждая из которых либо элементарна, либо функция на этой ступеньке равна нулю.
\end{to_def}  

\begin{to_thr}
\label{5.75}
    Пусть функция $f \colon \mathbb{R}^n \to \mathbb{R}$ интегрируема по Лебегу с конечным интегралом. Тогда $f$ можно сколь угодно близко приблизить в среднем элементарно ступенчатой функцией.
\end{to_thr}

\begin{to_thr}
\label{5.76}
    Пусть функция $f \colon X \to \mathbb{R} \in \L_c$. Положим для $M > 0$
    \begin{equation*}
    \begin{split}
        f_M(x) = \left\{
        \begin{aligned}
            &M,     &f(x) \geq M; \\
            &f(x)   &|f(x)| \leq M; \\
            &-M,    &f(x) \leq -M.
        \end{aligned}
        \right.
    \end{split}
    \end{equation*}
    Тогда 
    $$
    \lim_{M \to + \infty} \int_X f_M(x) \d x = \int_X f(x) \d x 
    \hspace{1cm} \text{и} \hspace{1cm} 
    \lim_{M \to + \infty} \int_X |f(x) - f_M(x)| \d x = 0.
    $$
\end{to_thr}