\sbs{Интеграл Лебега для ступенчатых и произвольных функций}

\begin{to_def}
    Функция $f \colon X \to \mathbb{R}$ называется \textbf{счётно ступенчатой}\footnote{
        Далее просто ступенчатая.
    }, если $X$ разбивается в счётное объединение измеримых множеств $X = \bigsqcup_i X_i$ и на каждом $X_i$ функция равна константе $c_i$.
\end{to_def}

% \noindent \textbf{Вопрос:} всякая ли ступенчатая функция измерима по Лебегу?

\begin{to_def}
    Для ступенчатой функции положим
    $$
    \int_X f(x) dx = \sum_i c_i \mu(X_i),
    $$
    считая $0 \cdot (+\infty) = 0$ и требуя, чтобы сумма абсолютно сходилась.
\end{to_def}

\begin{to_lem}
    Определение интеграла от ступенчатой функции корректно, а именно, если
функция является ступенчатой относительно двух разных разбиений области определения
на основания ступенек, $X = \bigsqcup_i X'_i = \bigsqcup_j X''_j$ ,то значение интеграла будет одним и тем же для обоих разбиений.
\label{5.58}
\end{to_lem}

\begin{to_lem}
    Для любой измеримой по Лебегу функции $f \colon X \to \mathbb{R}$ и любого $\varepsilon > 0 \; \exists$ ступенчатые $g, h \colon X \to \mathbb{R}$, такие что $g \leq f \leq h$ и 
    $$
    \int_X (h - g) \d x < \varepsilon.
    $$
    \label{5.59}
\end{to_lem}


\begin{to_def}
    Для измеримой по Лебегу функции $f \colon X \to \mathbb{R}$ определим \textbf{нижний интеграл Лебега} как
    $$
    \underline{\int_X} f(x) \d x = \sup \int_X g(x) \d x
    $$
    по интегрируемым ступенчатым $g \leq f$. Определим \textbf{верхний интеграл Лебега} как 
    $$
    \overline{\int_X} f(x) \d x = \inf \int_X h(x) \d x
    $$
    по интегрируемым ступенчатым $h \geq f$.
\end{to_def}

\begin{to_def}
    Функция $f \colon X \to \mathbb{R}$ \textbf{интегрируема по Лебегу с конечным интегралом}, если её нижний и верхний интеграл Лебега конечны и равны между собой.
\end{to_def}

\begin{to_lem}
    Если измеримая функция $f \colon X \to \mathbb{R}$ может быть оценена снизу и сверху, $g_0 \leq f \leq h_0$ ступенчатыми с конечным интегралом, то она сама имеет конечный интеграл Лебега.
    \label{5.62}
\end{to_lem}

\begin{to_def}
    Для неотрицатеьной измеримой функции $f \colon X \to \mathbb{R}^+$ будем писать $\int_X f(x) \d x = + \infty$, если её нижний интеграл Лебега бесконечен.
\end{to_def}

\begin{to_lem}
    Если неотрицательная измеримая $f \colon X \to \mathbb{R}^+$ имеет конечный нижний интеграл Лебега, то её верхний интеграл Лебега равен нижнему.
    \label{5.65}
\end{to_lem}

\begin{to_thr}
    Для измеримой $f \colon X \to \mathbb{R}$ положим
    \begin{equation*}
    \begin{split}
        f_+(x) = \left\{
        \begin{aligned}
            &f(x), &f(x) \geq 0; \\
            &0, &f(x) \leq 0; 
        \end{aligned}
        \right. \\
        f_-(x) = \left\{
        \begin{aligned}
            &f(x), &f(x) \leq 0; \\
            &0, &f(x) \geq 0.
        \end{aligned}
        \right. \\
    \end{split}
    \end{equation*}
    Интеграл $\int_X f(x) \d x$ определен и конечен тогда, и только тогда, когда интегралы $\int_X f_+ (x) \d x$ и $\int_X f_- (x) \d x$ определены и конечны.
    \label{5.66}
\end{to_thr}

\begin{to_thr}
    $f \colon [a,b] \to \mathbb{R}$ интегрируема по Риману $\Longleftrightarrow$ $f$ ограничена и $\mu(\{ \cdot \text{разрыва}\}) = 0$;
    \label{5.69}
\end{to_thr}