\sbs{\small Дифференцируемые отображения открытых подмножеств $\mathbb{R}^n$}

\begin{minipage}[]{0.45\textwidth}
%%%%%%%%%%%%%%%%%%%%%%%%%%%%%%%%%%%%%%%%%%%%%%%%%%%%%%%%%%%%%%%%%%%%%%%%
\begin{proof}[
lem $\triangle$
\eqref{6.3}]

\phantom{42}
\noindent

1)$\sum$ модулей координат $A x$ можно оценить как $\sum a_{i j} (\in A) \cdot |x_k|_{max}$;\\
2) Заметим $ |x_k|_{max} \leq \textbf{x} \leq \sum x_i$;\\
3) (2) $\Rightarrow$  $\sum a_{i j} (\in A)$ сгодится
в качестве $||A||$ в требуемом неравенстве.

\end{proof}


\begin{proof}[
thr $\triangle$
\eqref{6.10}]

\phantom{42}
\noindent
1) Дост. для $m=1$;\\
2) fix $x \in U$, р-м $\xi \in U_\delta (x) \subseteq U$. двигаясь по координатным осям можно дойти $x \to \xi$;\\
3) из (2): $f(\xi) - f(x) = f(\xi_1,..., \xi_n) -f(\xi_1,...,x_n)+...$;\\
4) по thr.Лагр: $f(\xi) - f(x) = \lr{\frac{\partial f}{\partial x_n}(x) + \oo(1)}(\xi_n - x_n) + ...$;\\
5) (4) и даёт непр. по опр., а $D f_x$ предст. в виде $\lr{\frac{\partial f_i}{\partial x_i}}$
\end{proof}
%%%%%%%%%%%%%%%%%%%%%%%%%%%%%%%%%%%%%%%%%%%%%%%%%%%%%%%%%%%%%%%%%%%%%%%%
\end{minipage}
\hfill
\begin{minipage}[]{0.45\textwidth}
%%%%%%%%%%%%%%%%%%%%%%%%%%%%%%%%%%%%%%%%%%%%%%%%%%%%%%%%%%%%%%%%%%%%%%%%
\begin{proof}[
 thr $\triangle$
\eqref{6.7}]

\phantom{42}
\noindent
1)$g(f(x)) =$
\begin{equation*}
	\begin{aligned}
		= &g(y_0) + D g_{y_0}(f(x) - f(x_0)) + \oo (|f(x) - f(x_0)|) = \\
		&= g(y_0) + D g_{y_0} \circ D f_{x_0} (x-x_0) + D g_{y_0} \oo (|x-x_0|) +\\
		&+ \oo (\text{O}(|x-x_0|)) = g(y_0) + D g_{y_0} \circ D f_{x_0}(x-x_0) + \oo (|x - x_0|)
	\end{aligned}
\end{equation*}
* Используя оценки из \eqref{6.3} $|A x| = \text{O}(|x|)$ и $f(x) - f(x_0) = \text{O}(x-x_0)$.
\end{proof}


\begin{proof}[
lem $\triangle$
\eqref{6.9}]

\phantom{42}
\noindent

1)Подставим $x + t v$ в опр. дифф-ла:\\
2) $f(x + t v) - f (x) = \d f_x (t v) + \oo (|t| |v|) = t (\d f_x(v) + \oo(1));$\\
3) Поделим на $t$ и перейдём к пределу.

\end{proof}
%%%%%%%%%%%%%%%%%%%%%%%%%%%%%%%%%%%%%%%%%%%%%%%%%%%%%%%%%%%%%%%%%%%%%%%%
\end{minipage}

