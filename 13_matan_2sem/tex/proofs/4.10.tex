\sbs{\small Интеграл Римана на отрезке и элементарно ступенчатые функции}

\begin{minipage}[t]{0.45\textwidth}
%%%%%%%%%%%%%%%%%%%%%%%%%%%%%%%%%%%%%%%%%%%%%%%%%%%%%%%%%%%%%%%%%%%%%%%%
\begin{proof}[
lem $\triangle$
\eqref{4.97}]

\phantom{42}
\noindent

1) Получ.разб.мельче: $\Delta = \Delta' \sqcup \Delta''$;\\
2) $\inf\{f(x) \mid x \in \Delta' \} \geq \inf \{ f(x) \mid x \in \Delta \}$\\
3) $\sup\{f(x) \mid x \in \Delta' \} \leq \sup \{ f(x) \mid x \in \Delta \}$\\
4) same для $\Delta''$, $|\Delta| = |\Delta'| + |\Delta''| \Rightarrow...$
\end{proof}


\begin{proof}[
 lem $\triangle$
\eqref{4.98}]

\phantom{42}
\noindent
1) из \eqref{4.97}:\\ $s(f,\tau) \leq s(f,\tau \vee \sigma) \leq S(f,\tau \vee \sigma) \leq S(f,\sigma).$
\end{proof}


\begin{proof}[
 lem $\triangle$
\eqref{4.108}]

\phantom{42}
\noindent

1) По опр. инт. Римана: оценим на $[a,b]$ с $\varepsilon$:\\$g\leq f\leq h$ и $\int_a^b (h - g) \d x < \varepsilon$;\\
2) но тогда, $\int_c^d (h - g) \d x < \varepsilon$, что и озн. 

\end{proof}




%%%%%%%%%%%%%%%%%%%%%%%%%%%%%%%%%%%%%%%%%%%%%%%%%%%%%%%%%%%%%%%%%%%%%%%%
\end{minipage}
\hfill
\begin{minipage}[]{0.45\textwidth}
%%%%%%%%%%%%%%%%%%%%%%%%%%%%%%%%%%%%%%%%%%%%%%%%%%%%%%%%%%%%%%%%%%%%%%%%
\begin{proof}[
 def $\triangle$
\eqref{RIMAN}]

\phantom{42}
\noindent

Ступ. функ:\\
\textbf{I)} Монотонность:\\
1)$f, g$ ступ. на $\tau, \sigma$, тогда они ступ. на $\varphi = \tau \vee \sigma$; 
2) Из адд. длины промежутка: инт. не изменится при измельчении.\\
\textbf{II)} Линейность:
1) опять $\varphi$, а на одном разбиении лин. очевидна.\\

Инт. Римана:\\
\textbf{I)} Монотонность: по опр. сумм Дарбу.\\
\textbf{II)} Линейность:\\
1) $h_f \leq f \leq H_f$, $h_g \leq g \leq H_g$, где $\int H-g < \varepsilon$;\\
2) $\int_a^b ((A H_f + B H_g) - (A h_f + B h_g)) \leq (A+B) \varepsilon$;\\
3) тогда огранив $A f + B g$ ступ, и проинт, получим схождение.

\textbf{III)} Аддитивность по отр.\\
1) Пусть на $[a,b]:$ $g_1 \leq f \leq h_1$;\\
2) на $[b,c]:$ $g_2 \leq f \leq h_2$: $\int_b^c (h_2-g_2)\ d x < \varepsilon$;\\
3) конкат: $g_1 + g_2 = g$, $h... \leadsto$ $g\leq f \leq h$;\\
4) $\int_a^c (h - g) \d x < 2 \varepsilon$, зажимающие $f$...

\end{proof}
%%%%%%%%%%%%%%%%%%%%%%%%%%%%%%%%%%%%%%%%%%%%%%%%%%%%%%%%%%%%%%%%%%%%%%%%
\end{minipage}