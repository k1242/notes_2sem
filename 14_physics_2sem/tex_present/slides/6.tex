\section{Странный аттрактор}

\begin{frame}
\frametitle{Конечное число «опасных» мультипликаторов}

Рассмотрим $\mu \sim 1$: card$\{\mu_i\} < + \infty$. Получим систему вида
$$
    \dot{\vc{x}}(t) = \vc{F}(\vc{x}).
$$
Система диссипативна, так что:
$$
    \text{div } \dot{\vc{x}}(t) = \text{div } \vc{F}(\vc{x}) \equiv \partial F^{(i)} / \partial x^{(i)}.
$$



\end{frame}
%%%%%%%%%%%%%%%%%%%%%%%%%%%%%%%%%%%%%%%%%%%%%%%%%%%%%%%%%%%%%%%%%%%%%%%%
\begin{frame}
\frametitle{«Странность» аттрактора}
Стремиться к предельному циклу или к незамкнутой намотке на торе? А может быть и по-другому..

\begin{itemize}
    \item[$\checkmark$] Рассмотрим внутри ограниченного объема траектори в предположении, что \textbf{они все неустойчивы}. 
    \item[$\checkmark$] Две сокль угодно близкие точки пространства состояний в дальнейшем разойдутся. 
    \item[$\checkmark$] Незамкнутая траектория может подойти к самой себе
сколь угодно близко. 
\end{itemize}

$\leadsto$ \textbf{турбулентное движение жидкости.}


\end{frame}
%%%%%%%%%%%%%%%%%%%%%%%%%%%%%%%%%%%%%%%%%%%%%%%%%%%%%%%%%%%%%%%%%%%%%%%%
\begin{frame}
\frametitle{Бифуркация разрушения квазипериодического режима}
    
Сколь угодно малая нелинейность может \textbf{разрушить намотку на торе}, создав на торе \textbf{странный аттрактор} \textit{(D. Ruelle, F. Tokens, 1971)}.

\phantom{42}

Но, по теореме Пуанкаре-Бендиксона  на двумерной поверхности
невозможно существование хаотического режима.

\phantom{42}

На третьей бифуркации возникновение странного аттрактора становится возможным \textit{(D. Ruelle, F. Tokens, 1978)}.
\end{frame}
%%%%%%%%%%%%%%%%%%%%%%%%%%%%%%%%%%%%%%%%%%%%%%%%%%%%%%%%%%%%%%%%%%%%%%%%
\begin{frame}
\frametitle{Структура (канторовость) странного аттрактора}
    
\begin{itemize}
    \item[$\checkmark$] Элемент объёма в окрестности седловой траектории в одном из направлений растягивается, в другом сжимается. 
    \item[$\checkmark$] Ввиду диссипативности объёмы должны уменьшаться. 
    \item[$\checkmark$] По ходу траекторий направления должны меняться.
\end{itemize}

$\Rightarrow$ уменьшение по площади и изгиб формы.

$\Rightarrow$ система вложенных полос, разделенных пустотами.

$\Rightarrow$ канторовость странного аттрактора (характерное свойство).

\end{frame}