
\begin{frame}\frametitle{Определения}

\begin{description}
    \item[Динамическая система (DS)] ---  тройка $\{E, f, T\}$, где 
        $\vc{E}$ -- фазовое пространство,  
        $\vc{T}$ -- множество, характеризующее эволюцию системы во времени;
        $\vc{f}$ -- дифференцируемеое отображение $T \times E \to E$, или $f^{t}(x)$, где $t \in T$.
    \item[Пространство состояний (фазовое пространство)] --- каждая точка овтечает распределению скоростей в ней. 
    \item[Аттрактор] ---  $\vc{E}$:  все траектории из $B \subset \vc{E}$ стремятся к нему, при $t \to \infty$.
    \item[Бифуркация] --- качественное изменение фазового портрета, при плавном изменении параметров DS. 
    
\end{description}

\end{frame}

\begin{frame}
\frametitle{Определения}

\begin{description}
    \item[Предельный цикл] --- ЗПТ системы дифференциальных уравнений, изолированная от других ЗПТ \textbf{и} ЗПТ: $\forall$ траекторий из некоторой окрестноти периодических траекторий стремится к ней при $t \longrightarrow + \infty$ (установившийся периодический цикл) \textbf{или} при $t \longrightarrow - \infty$ (неустановившийся предельный цикл).
    \item[Хаотичное отображение] --- $f$ \textbf{хаотично}, если \\
    1) Периодические точки всюду плотны в $\vc{E}$. \\
    2) Орбиты перемешиваются (почти): \\
     $U_1$, $U_2$ $\subset \vc{E}$. $ \forall x_0 \in U_1 \; \exists N \in \mathbb{N}: f^N (x_0) \in U_2. $ \\
    3) $f$ чувствительно к н.у.: \\
    $\forall x_0 \in \vc{E}, \; \forall U_{\varepsilon}(x_0) \; \exists y_0 \in U_{\varepsilon}, \exists N\in \mathbb{N} \colon |f^n(x_0) - f^n(y_0)| > \beta.$  
\end{description}

\end{frame}