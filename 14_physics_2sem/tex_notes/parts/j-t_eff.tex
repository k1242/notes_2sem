\section{Адиабатические процессы в газах}

\noindent
\textbf{Дросселирование газа} 

Считая $P_1$ и $P_2$ постоянными,получим, что работа в левой части $A_1 = P_1 V_1$,  в правой части $A_2 = P_2 V_2$. Тогда при $\delta Q = 0$,
$$
A_1 - A_2 =  \left(
U_2 + \dots
\right) - 
\left(
U_2 + \dots
\right).
$$
Таким образом $H_1 = U_1 + P_1 V_1 = H_2$. 

\phantom{42}

\noindent
\textbf{Уравнение Бернулли}

Вдоль линии тока:
\begin{equation}
    u + \frac{P}{\rho} + g h + \frac{v^2}{2} = \text{const}.
\end{equation}
Или:
\begin{equation}
\label{eq1}
   i  + g h + \frac{v^2}{2} = \text{const},
\end{equation}
где $i$ - удельная энтальпия.

\phantom{42}

\noindent
\textbf{Скорость истечения газа из отверстия}

Применим уравнение Бернулли:
$$ i_1 + \frac{v_1^2}{2} =  i_2 + \frac{v_2^2}{2} $$
Считая скорость $v_1$ внутри баллона пренебрежимо малой:
$$v_2 = \sqrt{2(i_1 - i_2)} $$
Допустим теперь, что газ идеальный и что зависимостью $C_V$ от температуры можно пренебречь.Значит:
$$ i = \frac {C_p T}{\mu}$$
Подставив это в выражение для $v$ и выразив $T$ через $P$ через уравнение адиабаты, получим в итоге:
\begin{equation}
    v_2 = \sqrt {\frac{2}{\mu} C_P T_1\left[1 - \left( \frac{P_2}{P_1} \right)^{\frac{\gamma - 1}{\gamma}}\right] \phantom{9}}
\end{equation}
% Мяяяяу!<3)

\phantom{42}

\noindent
\textbf{Дифференциальный эффект Джоуля Томпсона} 

Рассмотрим энтальпию, как функцию температуры и давления. Отсюда найдём $\ptdr{T}{P}{H}$. Далее используем $\ptdr{H}{T}{P}=C_P$, $\ptdr{H}{P}{T} = T \ptdr{S}{P}{T} + V = - T \ptdr{V}{T}{P} + V$. Таким образом, получаем\footnote{там не совчем частная, а $\Delta$}
\begin{equation}
\boxed{
    \label{jte}
    \ptdr{T}{P}{H} = \frac{1}{C_P} \left(
    T \ptdr{V}{T}{P} - V
    \right).
}
\end{equation}

Тогда для газа Ван-Дер-Ваальса:
\begin{equation}
\boxed{
    \ptdr{T}{P}{H} = \frac{1}{C_P \ptdr{P}{V}{T}} \left(
    \frac{b R T}{(V - b)^2} - \frac{2a}{V^2}
    \right)
    }
\end{equation}
Знак эффекта меняется при \textit{температуре инверсии}:
\begin{equation}
    T_{inv} = \frac{2a}{Rb} \frac{(V-b)^2}{V^2}.
\end{equation}

\phantom{42}

\noindent
\textbf{Интегральный эффект Джоуля Томпсона} \\
Газ в состоянии 2 разреженный (т.е. почти идеальный газ). Соответсвенно запишем сохранение энтальпии, подставим (\ref{vdv}), уравнение и.г., и получим:
\begin{equation}
    (C_V + R) (T_2 - T_1) = \frac{bRT_1}{V_1-b} - \frac{2a}{V_1}.
\end{equation}

Тогда температура инверсии:
\begin{equation}
    T_{inv} = \frac{2a}{Rb} \frac{V_1-b}{V_1}.
\end{equation}
