\section{Второе начало теродинамики}

\noindent
\textbf{Формулировка Клаузиуса.} Невозможен круговой процесс, \textit{единственным} результатом которого был бы переход тепла от тела более холожного к телу более нагретому (невозможна машина Клаузиуса). 

\phantom{42}

\noindent
\textbf{Формулировка Томпсона (Кельвина).} Невозможен круговой процесс, \textit{единтсвенным} результатом которого было бы совершение работы за счёт теплоты, взятой от одного какого-либо тела (невозможна машина Томсона).

\phantom{42}

\noindent
\textbf{THR.} Формудировки эквивалентны.

\begin{proof}[$\triangle$]
Пусть существует машина Томпсона. Преобразуем её работу в тепло переданное термостату с более высокой температурой. Получили машину Клаузиуса.  

Пусть существует машина Клаузиуса. Пусть есть 2 резервуара: $T_2 < T_1$. Запустим тепловую машину, пусть она произвела работу $A = Q_1 - Q_2$. Далее машиной Клаузиуса перегоним тепло $Q_2$ от холодильника к нагревателю. Получили машину Томпсона.   
\end{proof}

\phantom{42}

\noindent
\textbf{Теорема$_1$ Карно.}  КПД любых идеальных машин, работающих по циклу Карно между двумя заданными термостатами, равны и не зависят от конкретного устройства машин и вида рабочего тела.

\phantom{42}

\noindent
\textbf{Теорема$_2$ Карно.}  КПД любой тепловой машины, работающей между двумя резервуарами, не может превышать КПД машины Карно, работающей между теми же резервуарами.

\begin{proof}[$\triangle_1$]
Пусть это не так. Запустим две машины Карно K$_a$ и К$_b$: $\nu_1 > \nu_2$. Есть два резервуара Р$_1$ и Р$_2$: $T_1 > T_2$. Пусть К$_a$: $1 \to 2$, К$_b$: $2 \to 1$, использует работу $A_b$. Сделаем так, чтобы $A_a = A_b$. Тогда
$$
Q_{b1} = \frac{\eta_a}{\eta_b} Q_a1.
$$
Тогда Р$_1$ получит $\Delta Q_1$,  Р$_2$ потеряет $\Delta Q_2 = \Delta Q_1$
$$
\Delta Q_1 = \frac{\eta_a - \eta_b}{\eta_b} Q_1 > 0, \; \; 
\Delta Q_2 = (1 - \eta_b)Q_{b1} - (1-\eta_a)Q_{a1} = \Delta Q_1 >0.
$$
Противоречие с вторым началом. $\Rightarrow \eta_a \leqslant \eta_b$, но поменяв местами K$_a$ и К$_b$ получим, что  $\eta_a = \eta_b$.


\end{proof}
\begin{proof}[$\triangle_2$]
Аналогично предыдущему доказательству запустим машину Карно К $2 \to 1$ и более эффективную машину М $1 \to 2$. Аналогично придём к $\eta_M \leqslant \eta_K$.
\end{proof}

\phantom{42}

\noindent
\textbf{Неравенство Клаузиуса.} Учитывая, что равенство возможно только для обратимого процесса:
\begin{equation}
    \oint \frac{\delta Q}{T} \leqslant 0.
\end{equation}

\begin{proof}[$\triangle$]
По второй теореме Карно $\nu \leqslant \nu_K $, тогда $1 + Q_2/Q_1 \leqslant 1 - T_2/T_1$. Частный случай:
$$
\frac{Q_1}{T_1} + \frac{Q_2}{T_2}  \leqslant 0.
$$
Общо рассмотрим контакт с $n$ термостатов температуы $T_i$. Есть $T_0$ резервуар  и $n$ машин  Карно, осуществляющих перекачку тепла $Q_{0i}$. Для каждой машины
$$
\frac{Q_{0i}}{T_)} + \frac{Q_i'}{T_i} = 0
$$
Далее подбираются теплоты $Q_i'$ так чтобы полностью компенсировать "расход" $T_i$. Тогда $Q_0 = T_0 \sum_{i=1}^{n} Q_i / T_i$. Формально из адиабатичности системы $A = Q_0$, но \textit{по второму началу термодинамики} $A \leqslant 0$. Далее рассматривается бесконечно много разурвуаров. 
\end{proof}

\phantom{42}

\noindent 
\textbf{Энтропия.} Рассматривается обратимый круговой процесс из 1 в 2, через неравенсто Клаузиуса:
\begin{equation}
    S_2 - S_1 = \int_{1 \to 2} \frac{\delta Q}{T}
\end{equation}
В замкнутой системе энтропия не убывает.












