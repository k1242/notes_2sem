\section{Термодинамические потенциалы}

\subsection{Уравнение Майера}
\begin{equation}
\label{cp-cv}
    C_P - C_V = T \ptdr{P}{T}{V} \ptdr{V}{T}{P} = - T \frac{\ptdr{V}{T}{P}^2}{\ptdr{V}{P}{T}}
\end{equation}
Стоит заметить, что 
\begin{equation}
    C_P > C_V > 0.
\end{equation}

\begin{proof}
Распишем $U(T, V)$. 
    \begin{equation*}
    C dT = dU + PdV = \ptdr{U}{T}{V}dT + \left(
    \ptdr{U}{V}{T} + P
    \right) dV.
\end{equation*}
Учитывая, что
\begin{equation}
\label{derU}
    \ptdr{U}{V}{T} = T \ptdr{S}{V}{T} - P = T \ptdr{P}{T}{V} - P,
\end{equation}
получим, Q.E.D.
\end{proof}

\begin{table}[h]
\caption{Термодинамические потенциалы}
    \centering
    \begin{tabular}{c|ll}
    \toprule
    %  name 1 &  name 2  \\
    % \midrule
        $U_{(S, V)}$ &                   & $dU = +TdS - PdV$      \\
        $H_{(S, P)}$ & $= U + PV$        & $dH = +TdS + VdP$      \\
        $F_{(T, V)}$ & $= U - TS$        & $dF = -SdT - PdV$      \\
        $G_{(T, P)}$ & $= U + PV - TS$   & $dG = -SdT + VdP$      \\
    \bottomrule
    \end{tabular}
    \label{tab1}
\end{table}

\subsection{Условия термодинамической устойчивости}
Рассмотрим систему тело $\left(P, T, V \right)$ $+$ термостат $\left( P_0, V_0, T_0 \right)$. Из неравенства Клаузиуса:
$$
0 = dU + \delta A - \delta Q = dU + P_0 dV - \delta Q  \geqslant dU + P_0 dV - T_0 dS \equiv dZ,
$$
где введено обозначение $Z = Z(V, S) = U + P_0 V - T_0 S$.

\phantom{42}

\noindent
\textbf{Экстремальность}:  Частные производные по $V$ и $S$ ноль. Считая процесс квазистатическим находим, что $P=P_0$, $T = T_0$. 

\phantom{42}

\noindent
\textbf{Минимальность}: Зная, что $d^2 Z \geqslant 0$, получаем $d^2 U \geqslant 0$, что доводит до:
\begin{align}
\label{a}
    \left( \frac{ 
    \partial^2 U
    }{
    \partial S^2
    } \right)_{V} &> 0 \\
    X \equiv
    \label{b}
    \left( \frac{ 
    \partial^2 U
    }{
    \partial S^2
    } \right)_{V}
    \left( \frac{ 
    \partial^2 U
    }{
    \partial V^2
    } \right)_{S} -
    \left( \frac{ 
    \partial^2 U
    }{
    \partial S \partial V
    } \right)^2 &> 0.
\end{align}

Из (\ref{a}), получим:
\begin{equation*}
    \left( \frac{ 
    \partial^2 U
    }{
    \partial S^2
    } \right)_{V} = \ptdr{T}{S}{V} =  \frac{T}{C_V} > 0, \text{ т. е. } 
    \boxed{C_V > 0}.
\end{equation*}

Из (\ref{b}), получим:
\begin{align*}
    X = - \ptdr{T}{S}{V} \ptdr{P}{V}{S} + \ptdr{T}{V}{S} \ptdr{P}{S}{V} > 0, \\
    \text{далее рассмотрим $P \equiv P\left(V, T\right)$} \\
    \ptdr{P}{V}{S} = \ptdr{P}{V}{T} + \ptdr{P}{T}{V} \ptdr{T}{V}{S}. \\
    \text{Из последних двух уравнений получим:} \\
    \boxed{X = - \frac{T}{C_V} \ptdr{P}{V}{T} > 0}.
\end{align*}

Уже зная, что $C_V > 0,$ получаем, что $\boxed{\ptdr{P}{V}{T}<0}$. Пожинаем плоды:  $\beta_T > 0$, ${C_P > C_V}$, $\gamma > 1$.


