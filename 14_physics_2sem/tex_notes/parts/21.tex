\sec{Распределение Больцмана}

Для молекул газа, описываемых \textbf{классической механикой}, можно ввести распределение молекул по \textbf{фазовому пространству}.
Пусть $\d N$ -- среднее число молекул, c $r$ степенями свободы, в элементе объёма $\d p \d q = \d p_1 ... \d p_r \d q_1 ... \d q_r$. Или: 
\begin{equation*}
 \d N = n(p, q) \d \tau, \hspace{1cm} \tau = \frac{\d p \d q}{(2 \pi \hbar)^r}, \hspace{1cm} n(p,q) = \exp \frac{\mu - \varepsilon(p,q)}{T}
\end{equation*}

Для поля внешних сил, введём объемную силу $f$ (на одну частицу). Для слоя газа толщиной $\d z$, площадью S, в элементе объёма $\d N = n S \d z$ частиц. Условие механического равновесия для них:

\begin{equation*}
	f_z \d N + P(z) \d S - P(z \d z) S = 0 \leadsto \frac{\partial P}{\partial z} = f_z n \\	
\end{equation*}

Откуда получаем барометрическую формулу, \textit{aka} распределение Больцмана: $\boxed{n = n_0 \exp{\lr{-\frac{mgh}{kT}}}}$