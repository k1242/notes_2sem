\section*{Абстракт}
\addcontentsline{toc}{section}{Абстракт}

В настоящей работе обозревается проблема турбулентности. На основе $\S 30, \S 31, \S 32$ из \cite{LL6} и \cite{HS}, рассмотрены некоторые сценарии перехода к турбулентности. Разобран сценарий Ландау-Хопфа и рассмотрен эффект синхронизации колебаний. На основе \cite{Kuz} решена \textbf{задача о конвекции в замкнутой петле}. Для развития темы \textbf{странного аттрактора} совершена попытка повторить эксперимент с водяным колесом из \cite{water_wheel}, но всвязи с техническими трудностями (установка сломалась до снятия данных), результат ограничен численным моделированием системы в \textit{Python}. Построен аттрактор Лоренца. В рамках развития темы \textbf{сценария удвоения периода} в \textit{Wolfram Mathematica} построена модель логистического отображения. Рассчитана зависимость показателя Ляпунова от параметра. Аналогично проанализировано отображение Гаусса.