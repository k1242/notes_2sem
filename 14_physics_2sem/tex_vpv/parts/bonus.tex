% 

\section{Из теории динамических систем}

\subsection{Теорема Пуанкаре-Бендиксона}
\begin{to_thr}
Пусть выполняются следующие условия:
\begin{enumerate}
    \item $X$ -- замкнутое, ограниченное множество в $E^2$.
    \item DS задаёт непрерывно-дифференцируемое векторное поле на открытом множестве, содержащем $R$.
    \item $X$ не содержит неподвижных точек ($\dot{\vc{x}} \neq 0$).
    \item Существует траектория $S \subseteq X$ .
\end{enumerate}
Тогда $S$ -- либо \textbf{замкнутая траектория},
либо \textbf{накручивается на замкнутую траекторию}\footnote{
Что соответсвует существованию \textbf{предельного цикла}.
}. В любом случае $X$ содержит замкнутую орбиту.
\end{to_thr}


\subsection{Определение динамического хаоса}
Отображение $f$ \textbf{хаотично}, если \\
    \phantom{42} \hspace{0.5cm} 1) Периодические точки всюду плотны в $\vc{E}$. \\
    \phantom{42} \hspace{0.5cm} 2) Орбиты перемешиваются (почти):
    пусть $U_1$, $U_2$ $\subset \vc{E}$. $ \forall x_0 \in U_1 \; \exists N \in \mathbb{N}: f^N (x_0) \in U_2. $ \\
    \phantom{42} \hspace{0.5cm} 3) $f$ чувствительно к н.у.:
    $\forall x_0 \in \vc{E}, \; \forall U_{\varepsilon}(x_0) \; \exists y_0 \in U_{\varepsilon}, \exists N\in \mathbb{N} \colon |f^n(x_0) - f^n(y_0)| > \beta\footnote{
    $\beta$ -- константа чувствительности.
    }.$  


\subsection{Отображение Пуакаре}
Пусть $\vc{T} = \mathbb{R}$. Расположим\footnote{
Так, чтобы интересующие фазовые траектории многократно пересекали $S$ и были бы исключены касания.
}
в фазовом пространстве двумерную площадку $S$ и зададим на ней некоторую систему координат $(X, Y)$.

Так получается динамическая система $\{S, f_N, T_N\}$, где 
$T_N \subseteq \mathbb{N}$.  Функция $f_N$ такая, что 
$f_N(r) = \{f^t(r) \mid \min t \neq 0 : f^t(r) \in S  \}$. \textbf{Отображение Пуанкаре:}
\begin{align*}
    x_{n+1} &= f_N(x_n, y_n)_x, \\
    y_{n+1} &= f_N(x_n, y_n)_y.
\end{align*}
Именно через \textit{сечение Пуанкаре} можно анализировать становление регулярного или хаотического режима. 


\subsection{Устойчивость по линейному приближению}
Рассмотрим устойчивость по линейному приближению системы: $\dot{\vc{x}} = \vc{A}(\vc{x})$, $\dot{\vc{x}} = 0$ -- положение равновесия и разложим $\vc{A}(x)$:
\begin{equation}
    \dot{\vc{x}} = \vc{A}(0) + \dfrac{\partial \vc{A}(0)}{\partial \vc{x}^T} \vc{x} + \dots \approx \dfrac{\partial \vc{f}(0)}{\partial \vc{x}^T} \vc{x} = \mathbb{J} \vc{x}.
\end{equation}

% \begin{to_thr}[\textbf{Ляпунова об устойчивости по линейному приближению}]
%     $\mathbb{J} = const$, для собственного числа $\lambda$: $|\mathbb{J - \lambda E}| = 0$, собственные векторы $\--$ $\lambda_1, \dots, \lambda_n$: \\
%     \phantom{42} \hspace{1cm} 1) Если все $\lambda_i$ матрицы $\mathbb{J}$ по линейному приближению:
%         $\Re{\lambda_i} < 0,$ то соответственно $\circledast$ и линейное приближение асимптотически устойчиво. \\
%     \phantom{42} \hspace{1cm} 2) $\exists \lambda_i: \Re{\lambda_i} > 0$, то соответственно положение равновесия системы и линейное приближение неустойчиво.
% \end{to_thr}

\subsection{Экспонента Ляпунова}
Определим \textbf{показатель Ляпунова} как 
    $\lambda (f, x) = \lim_{n \to \infty} \frac{1}{n} \log{|(f^n)' (x)|},$
где $(f^n)'$ -- производная nной итерации. Возможно, что предел не сущствует. Тогда в зависимости от ваших нужд можете рассмотреть $\varliminf$ или $\varlimsup$. Корректнее рассматривать $\varliminf$, но за $\varlimsup$ вас не осудят.

Можно проще. Воспользуемся тем, что 
$(f^n)'(x) = f'(x_{n-1}) \dots f'(x_1) f'(x_0)$, тогда
\begin{equation}
    \boxed{\lambda (f, x) - \lim_{n \to \infty} \frac{1}{n} \sum_{k=0}^{n-1} \log |f'(x_k)|.} 
\end{equation}
