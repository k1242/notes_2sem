\section{Странный аттрактор  \texorpdfstring{($\S 31$)}{Lg}}

Исчерпывающей теории возникновения турбулентности в различных типах гидродинамических течениях ещё не существует. 

\subsection{Конечное число <<опасных>> мультипликаторов}

При анализе устойчивости движения интересны мультипликаторы, по модулю близкие к 1. Для течения вязкой жидкости число таких опасных мультипликаторов \textbf{конечно}. Допускаемые различные возмущения обладают разными длинами расстояний, на которых существенно меняется скорость $\vc{v}_2$. Чем меньше масштаб движения, тем больше градиенты скорости в нем и тем сильнее оно тормозится вязкостью. Если расположить допустимые моды в порядке убывания их масштабов, то опасным может оказаться только некоторое конечное число из них. Таким образом выяснение возможных сценариев потери устойчивости может быть сведено к конечному числу переменных\footnote{
    Например, амплитуды компонент разложения поля скоростей в ряд Фурье по координатам.
}. Соотвественно переходим к конечномерному пространству состояний.

Речь об исследовании эволюции системы, описываемой уравнениями вида
\begin{equation}
    \dot{\vc{x}}(t) = \vc{F}(\vc{x}),
\end{equation}
где $\vc{x}(t)$ -- вектор в пространстве $n$ величин $x^{(1)}$, $x^{(2)}$, $\ldots$, $x^{(n)}$. Для диссипативной системы дивергенция $\dot{\vc{x}}$ в $\vc{x}$-пространстве отрицательна, что соотвествует сокращению объёмов $\vc{x}$-пространства при движении\footnote{
    По теореме Лиувилля для гамильтоновой системы дивергенция равна нулю.
}:
\begin{equation}
    \text{div } \dot{\vc{x}}(t) = \text{div } \vc{F}(\vc{x}) \equiv \partial F^{(i)} / \partial x^{(i)}
\end{equation}


\subsection{<<Странность>> аттрактора}

Явление синхронизации упрощает движение. Раннее подразумевалось, что при потере устойчивости периодическим движением возникает другое периодическое движение. \textit{Это не обязательно}. Ограниченность амплитуд пульсаций скорости обеспечивает лишь ограниченность объема пространства состояний. Траектории могут стремиться к предельному циклу, или к незамкнутой намотке на торе, но может быть и по-другому. 

Рассмотрим внутри ограниченного объема траектори в предположении, что они все неустойчивы. Две сокль угодно близкие точки пространства состояний в дальнейшем разойдутся (\textbf{см. показатель Ляпунова}). Ввиду огарниченности траектории незамкнутая траектория может подойти к самой себе сколь угодно близко. Именно такое нерегулярное поведение ассоциируется с турбулентным движением жидкости (\textbf{см. определение динамического хаоса}). 

Важный аспект такой системы -- чувствительность к начальным условиям. При неустойчивости движения исходная неточность со временем нарастает и дальнейшее состояние системы непредсказуемо.

Притягивающее множество неустойчивых траекторий в пространстве состояний может существовать (\textit{Э. Лоренц, 1963})\footnote{
    РАссмотрено движение воздушных потоков в плоском слое жидкости постоянной толщины при разложении скорости течения и температуры в двойные ряды Фурье с последующем усечением до первых-вторых гармоник.
}; его принято называть \textbf{странным аттрактором}.

Важно понимать, что траектории могут быть неустойчивыми по одним направлениям и устойчивыми по другим. Такие траектории называют седловыми и именно множество таких траекторий составляет странный аттрактор.

\subsection{Бифуркация разрушения квазипериодического режима}

Сколь угодно малая нелинейность может разрушить квазипериодический режим (намотку на торе), создав на торе странный аттрактор (\textit{D. Ruelle, F. Tokens, 1971}). На второй бифуркации, как уже говорилось, появляется незамкнутая обмотка на торе. \red{Учёт малой нелинейности не разрушает тора}, так что странный аттрактор должен был бы быть расположен на нем. Но на двумерной поверхности невозможно существование \textbf{хаотического режима} (\textbf{см. теорему Пуанкаре-Бендиксона}). 

\textbf{Пересечение в пространстве состояний траекторий противоречило бы причинности поведения классических систем.} Состояние системы в каждый момент времени однозначно определяет ее поведение в следующие моменты. На двумерной поверхности невозможность пересечений настолько упорядочивает поток траекторией, что его хаотизация невозможна. 

На третьей бифуркации возникновение странного аттрактора становится возможным (хотя и не обязательным). Такой аттрактор, приходящий на смену трёхчастотному периодическому режиму расположен на трёхмерном торе (\textit{D. Ruelle, F. Tokens, 1978}). 

\subsection{Структура (канторовость) странного аттрактора}

Некоторые суждения следуют из неустойчивости траекторий седлового типа и диссипативности системы.

Рассмотрим пучок траекторий на пути к аттрактору (переходный режим движения к установлению турбулентности). В поперечном сечении пучка траектории заполняют определенную площадь. 

Элемент объёма в окрестности седловой траектории в одном из направлений растягивается, в другом сжимается. Ввиду диссипативности объёмы должны уменьшаться. По ходу траекторий направления должны меняться. 

Это приводит к уменьшению по площади и изгибу формы. Это происходит с каждым элементом по площади. В результате сечение разбивается на систему вложенных полос, разделенных пустотами.

Возникающей в пределе $t \to \infty$ аттрактор представляет собой несчётное множество не касающихся друг друга слоёв --  поверхностей, на которых располагаются седловые траектории. Каждая из траекторий аттрактора блуждает по всем слоям и по прошествии достаточно большого времени пройдёт достаточно близко к любой точке аттрактора (свойство \textbf{эргодичности}). 

В математической терминологии такие множества относятся к категории \textbf{канторовых}. Именно канторовость (соотвественно, \textbf{самоподобие}) является наиболее характерным свойством аттрактора. 

\subsection{Показатель Ляпунова}

Ввиду эргодичности средние характеристики движения на странном аттракторе могут быть установлены путём анализа движения уже вдоль одной неустойчивой траектории в пространстве состояний. 

Пусть $\vc{x} = \vc{x}_0 (t)$ -- уравнение траектории. Рассмотрим деформацию <<сферического>> объёма при его перемещении вдоль этой траектории. Она определяется уравнениями, линеаризованными по разности $\vc{\xi} = \vc{x} - \vc{x}_0 (t)$ -- отклонение траекторий, соседних с данной. Эти уравнения имеют вид
\begin{equation}
    \dot \xi^{(i)} = A_{ik} (t) \xi^{(k)}, \hspace{0.5cm} A_{ik} (t) = \frac{\partial F^{(i)}}{\partial x^{(k)}} \bigg|_{x=x_0(t)}
\end{equation}
По мере движения длины осей ($l_s(t)$) эллипсоида меняются. \textbf{Ляпуновскими характеристическими показателями} называют предельные значения (\textbf{см. показатель Ляпунова}).